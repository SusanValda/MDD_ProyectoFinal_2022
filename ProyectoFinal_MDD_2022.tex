% Options for packages loaded elsewhere
\PassOptionsToPackage{unicode}{hyperref}
\PassOptionsToPackage{hyphens}{url}
%
\documentclass[
]{article}
\usepackage{amsmath,amssymb}
\usepackage{lmodern}
\usepackage{iftex}
\ifPDFTeX
  \usepackage[T1]{fontenc}
  \usepackage[utf8]{inputenc}
  \usepackage{textcomp} % provide euro and other symbols
\else % if luatex or xetex
  \usepackage{unicode-math}
  \defaultfontfeatures{Scale=MatchLowercase}
  \defaultfontfeatures[\rmfamily]{Ligatures=TeX,Scale=1}
\fi
% Use upquote if available, for straight quotes in verbatim environments
\IfFileExists{upquote.sty}{\usepackage{upquote}}{}
\IfFileExists{microtype.sty}{% use microtype if available
  \usepackage[]{microtype}
  \UseMicrotypeSet[protrusion]{basicmath} % disable protrusion for tt fonts
}{}
\makeatletter
\@ifundefined{KOMAClassName}{% if non-KOMA class
  \IfFileExists{parskip.sty}{%
    \usepackage{parskip}
  }{% else
    \setlength{\parindent}{0pt}
    \setlength{\parskip}{6pt plus 2pt minus 1pt}}
}{% if KOMA class
  \KOMAoptions{parskip=half}}
\makeatother
\usepackage{xcolor}
\usepackage[margin=1in]{geometry}
\usepackage{color}
\usepackage{fancyvrb}
\newcommand{\VerbBar}{|}
\newcommand{\VERB}{\Verb[commandchars=\\\{\}]}
\DefineVerbatimEnvironment{Highlighting}{Verbatim}{commandchars=\\\{\}}
% Add ',fontsize=\small' for more characters per line
\usepackage{framed}
\definecolor{shadecolor}{RGB}{248,248,248}
\newenvironment{Shaded}{\begin{snugshade}}{\end{snugshade}}
\newcommand{\AlertTok}[1]{\textcolor[rgb]{0.94,0.16,0.16}{#1}}
\newcommand{\AnnotationTok}[1]{\textcolor[rgb]{0.56,0.35,0.01}{\textbf{\textit{#1}}}}
\newcommand{\AttributeTok}[1]{\textcolor[rgb]{0.77,0.63,0.00}{#1}}
\newcommand{\BaseNTok}[1]{\textcolor[rgb]{0.00,0.00,0.81}{#1}}
\newcommand{\BuiltInTok}[1]{#1}
\newcommand{\CharTok}[1]{\textcolor[rgb]{0.31,0.60,0.02}{#1}}
\newcommand{\CommentTok}[1]{\textcolor[rgb]{0.56,0.35,0.01}{\textit{#1}}}
\newcommand{\CommentVarTok}[1]{\textcolor[rgb]{0.56,0.35,0.01}{\textbf{\textit{#1}}}}
\newcommand{\ConstantTok}[1]{\textcolor[rgb]{0.00,0.00,0.00}{#1}}
\newcommand{\ControlFlowTok}[1]{\textcolor[rgb]{0.13,0.29,0.53}{\textbf{#1}}}
\newcommand{\DataTypeTok}[1]{\textcolor[rgb]{0.13,0.29,0.53}{#1}}
\newcommand{\DecValTok}[1]{\textcolor[rgb]{0.00,0.00,0.81}{#1}}
\newcommand{\DocumentationTok}[1]{\textcolor[rgb]{0.56,0.35,0.01}{\textbf{\textit{#1}}}}
\newcommand{\ErrorTok}[1]{\textcolor[rgb]{0.64,0.00,0.00}{\textbf{#1}}}
\newcommand{\ExtensionTok}[1]{#1}
\newcommand{\FloatTok}[1]{\textcolor[rgb]{0.00,0.00,0.81}{#1}}
\newcommand{\FunctionTok}[1]{\textcolor[rgb]{0.00,0.00,0.00}{#1}}
\newcommand{\ImportTok}[1]{#1}
\newcommand{\InformationTok}[1]{\textcolor[rgb]{0.56,0.35,0.01}{\textbf{\textit{#1}}}}
\newcommand{\KeywordTok}[1]{\textcolor[rgb]{0.13,0.29,0.53}{\textbf{#1}}}
\newcommand{\NormalTok}[1]{#1}
\newcommand{\OperatorTok}[1]{\textcolor[rgb]{0.81,0.36,0.00}{\textbf{#1}}}
\newcommand{\OtherTok}[1]{\textcolor[rgb]{0.56,0.35,0.01}{#1}}
\newcommand{\PreprocessorTok}[1]{\textcolor[rgb]{0.56,0.35,0.01}{\textit{#1}}}
\newcommand{\RegionMarkerTok}[1]{#1}
\newcommand{\SpecialCharTok}[1]{\textcolor[rgb]{0.00,0.00,0.00}{#1}}
\newcommand{\SpecialStringTok}[1]{\textcolor[rgb]{0.31,0.60,0.02}{#1}}
\newcommand{\StringTok}[1]{\textcolor[rgb]{0.31,0.60,0.02}{#1}}
\newcommand{\VariableTok}[1]{\textcolor[rgb]{0.00,0.00,0.00}{#1}}
\newcommand{\VerbatimStringTok}[1]{\textcolor[rgb]{0.31,0.60,0.02}{#1}}
\newcommand{\WarningTok}[1]{\textcolor[rgb]{0.56,0.35,0.01}{\textbf{\textit{#1}}}}
\usepackage{longtable,booktabs,array}
\usepackage{calc} % for calculating minipage widths
% Correct order of tables after \paragraph or \subparagraph
\usepackage{etoolbox}
\makeatletter
\patchcmd\longtable{\par}{\if@noskipsec\mbox{}\fi\par}{}{}
\makeatother
% Allow footnotes in longtable head/foot
\IfFileExists{footnotehyper.sty}{\usepackage{footnotehyper}}{\usepackage{footnote}}
\makesavenoteenv{longtable}
\usepackage{graphicx}
\makeatletter
\def\maxwidth{\ifdim\Gin@nat@width>\linewidth\linewidth\else\Gin@nat@width\fi}
\def\maxheight{\ifdim\Gin@nat@height>\textheight\textheight\else\Gin@nat@height\fi}
\makeatother
% Scale images if necessary, so that they will not overflow the page
% margins by default, and it is still possible to overwrite the defaults
% using explicit options in \includegraphics[width, height, ...]{}
\setkeys{Gin}{width=\maxwidth,height=\maxheight,keepaspectratio}
% Set default figure placement to htbp
\makeatletter
\def\fps@figure{htbp}
\makeatother
\setlength{\emergencystretch}{3em} % prevent overfull lines
\providecommand{\tightlist}{%
  \setlength{\itemsep}{0pt}\setlength{\parskip}{0pt}}
\setcounter{secnumdepth}{5}
\ifLuaTeX
  \usepackage{selnolig}  % disable illegal ligatures
\fi
\IfFileExists{bookmark.sty}{\usepackage{bookmark}}{\usepackage{hyperref}}
\IfFileExists{xurl.sty}{\usepackage{xurl}}{} % add URL line breaks if available
\urlstyle{same} % disable monospaced font for URLs
\hypersetup{
  pdftitle={ProyectoFinal\_MDD},
  pdfauthor={Susan Valda - Andrew Serrano},
  hidelinks,
  pdfcreator={LaTeX via pandoc}}

\title{ProyectoFinal\_MDD}
\author{Susan Valda - Andrew Serrano}
\date{2022-11-18}

\begin{document}
\maketitle

{
\setcounter{tocdepth}{2}
\tableofcontents
}
\hypertarget{introducciuxf3n}{%
\section{Introducción}\label{introducciuxf3n}}

\hypertarget{motivaciuxf3n}{%
\section{Motivación}\label{motivaciuxf3n}}

\hypertarget{marco-teuxf3rico}{%
\section{Marco Teórico}\label{marco-teuxf3rico}}

\hypertarget{metodologuxeda}{%
\section{Metodología}\label{metodologuxeda}}

Para investigar las características asociadas a la práctica del trabajo
infantil se tomó como referencia la metodología KDD, por lo cual se
plantean los siguientes pasos: 1. Recolección de la información
referente a encuestas que contemplen la situación de
niños/niñas/adolescentes que realiacem algún tipo de trabajo infantil en
Bolivia. 2. Selección y tratamiento de los datos. 3. Utilizar la
librería RPART y CART de Rstudio para generar un modelo y árbol de
decisión en base a los datos seleccionados. 5. Finalmenta se efectúa el
análisis e interpretación del árbol de desición obtenido.

\hypertarget{datos}{%
\section{Datos}\label{datos}}

Se emplearon los datos de la Encuesta de niñas, niños y adolescentes
(ENNA) que realizan una actividad laboral o trabajan 2016 obtenida del
Instituto Nacional de Estadística (INE).

La base de datos ENNA cuenta con 10488 observaciones y 212 Variables, de
las cuales se utilizarán las siguientes variables para la elaboración
del árbol de decisión.

\begin{longtable}[]{@{}
  >{\raggedright\arraybackslash}p{(\columnwidth - 2\tabcolsep) * \real{0.0803}}
  >{\raggedright\arraybackslash}p{(\columnwidth - 2\tabcolsep) * \real{0.9197}}@{}}
\toprule()
\begin{minipage}[b]{\linewidth}\raggedright
Código
\end{minipage} & \begin{minipage}[b]{\linewidth}\raggedright
Variable
\end{minipage} \\
\midrule()
\endhead
depto & Departamento \\
area & Urbana Rural \\
ns001a\_02 & Sexo \\
ns001a\_03 & ¿Cuántos años cumplidos tienes? \\
ns01a\_01 & ¿Sabes leer y escribir? \\
ns01a\_02a & ¿Cuál fue el último NIVEL Y CURSO más alto que aprobaste?
NIVEL O CICLO \\
ns01a\_02b & ¿Cuál fue el último NIVEL Y CURSO más alto que
aprobaste?CURSO O GRADO \\
ns01a\_03 & Durante este año ¿Estás o estuviste inscrito en algún curso
o grado de educación escolar, alternativa o superior? \\
ns01a\_05c & ¿En qué turno te has inscrito este año (2016)? \\
ns01a\_06 & ¿Asistes/Asististe regularmente al curso al que te has
inscrito este año (2016)? \\
ns03a\_04 & ¿Estás de acuerdo con realizar estas tareas domésticas? \\
ns04a\_02 & Durante la semana pasada, ¿consideras que tuviste un tiempo
adecuado para descanso(relajación, ocio sano) o recreación\ldots? \\
ns03a\_01a & Durante la semana pasada, ¿realizaste para este hogar,
alguna de las tareas domésticas indicadas a continuación? (Tarea N°1) \\
ns03a\_01b & Durante la semana pasada, ¿realizaste para este hogar,
alguna de las tareas domésticas indicadas a continuación? (Tarea N°2) \\
ns03a\_01c & Durante la semana pasada, ¿realizaste para este hogar,
alguna de las tareas domésticas indicadas a continuación? (Tarea N°3) \\
ns03a\_01d & Durante la semana pasada, ¿realizaste para este hogar,
alguna de las tareas domésticas indicadas a continuación? (Tarea N°4) \\
ns03a\_01e & Durante la semana pasada, ¿realizaste para este hogar,
alguna de las tareas domésticas indicadas a continuación? (Tarea N°5) \\
ns03a\_01f & Durante la semana pasada, ¿realizaste para este hogar,
alguna de las tareas domésticas indicadas a continuación? (Tarea N°6) \\
ns03a\_01g & Durante la semana pasada, ¿realizaste para este hogar,
alguna de las tareas domésticas indicadas a continuación? (Tarea N°7) \\
ns03a\_01h & Durante la semana pasada, ¿realizaste para este hogar,
alguna de las tareas domésticas indicadas a continuación? (Tarea N°8) \\
ns03a\_03a & Durante la semana pasada, ¿realizaste estas tareas
usualmente\ldots? \\
\bottomrule()
\end{longtable}

La variable que ayudará a entrenar el modelo es:

\begin{longtable}[]{@{}ll@{}}
\toprule()
Código & Variable \\
\midrule()
\endhead
condac & Condición de actividad \\
\bottomrule()
\end{longtable}

\hypertarget{recopilaciuxf3n-de-la-fuente-de-datos}{%
\subsection{Recopilación de la fuente de
datos}\label{recopilaciuxf3n-de-la-fuente-de-datos}}

\hypertarget{base-de-datos}{%
\subsubsection{Base de datos}\label{base-de-datos}}

\begin{Shaded}
\begin{Highlighting}[]
\FunctionTok{library}\NormalTok{(haven)}\CommentTok{\#para importar bases de datos}

\CommentTok{\#Carga de datos}
\FunctionTok{load}\NormalTok{(}\FunctionTok{url}\NormalTok{(}\StringTok{"https://github.com/AlvaroLimber/EST{-}384/blob/master/data/nna16.RData?raw=true"}\NormalTok{))}
\end{Highlighting}
\end{Shaded}

\hypertarget{selecciuxf3n}{%
\subsection{Selección}\label{selecciuxf3n}}

\hypertarget{pre-procesado}{%
\subsection{Pre-procesado}\label{pre-procesado}}

\begin{Shaded}
\begin{Highlighting}[]
\FunctionTok{library}\NormalTok{(dplyr)}
\end{Highlighting}
\end{Shaded}

\begin{verbatim}
## 
## Attaching package: 'dplyr'
\end{verbatim}

\begin{verbatim}
## The following objects are masked from 'package:stats':
## 
##     filter, lag
\end{verbatim}

\begin{verbatim}
## The following objects are masked from 'package:base':
## 
##     intersect, setdiff, setequal, union
\end{verbatim}

\begin{Shaded}
\begin{Highlighting}[]
\FunctionTok{library}\NormalTok{(tidyr)}
\FunctionTok{library}\NormalTok{(Hmisc)}
\end{Highlighting}
\end{Shaded}

\begin{verbatim}
## Loading required package: lattice
\end{verbatim}

\begin{verbatim}
## Loading required package: survival
\end{verbatim}

\begin{verbatim}
## Loading required package: Formula
\end{verbatim}

\begin{verbatim}
## Loading required package: ggplot2
\end{verbatim}

\begin{verbatim}
## 
## Attaching package: 'Hmisc'
\end{verbatim}

\begin{verbatim}
## The following objects are masked from 'package:dplyr':
## 
##     src, summarize
\end{verbatim}

\begin{verbatim}
## The following objects are masked from 'package:base':
## 
##     format.pval, units
\end{verbatim}

\begin{Shaded}
\begin{Highlighting}[]
\NormalTok{bd }\OtherTok{\textless{}{-}}\NormalTok{ nna }\SpecialCharTok{\%\textgreater{}\%} \FunctionTok{select}\NormalTok{(condac,depto,area,}\AttributeTok{sexo=}\NormalTok{ns001a\_02,}\AttributeTok{edad=}\NormalTok{ns001a\_03,}\AttributeTok{sabeLeer=}\NormalTok{ns01a\_01,}
\AttributeTok{maxNivelAprobado=}\NormalTok{ns01a\_02a,}\AttributeTok{maxCursoAprobado=}\NormalTok{ns01a\_02b,}\AttributeTok{inscAlternativa=}\NormalTok{ns01a\_03,}\AttributeTok{inscrito=}\NormalTok{ns01a\_05c,}
\AttributeTok{asistencia=}\NormalTok{ns01a\_06,}\AttributeTok{tareasDomesticas=}\NormalTok{ns03a\_04,}\AttributeTok{tiempoDescanso=}\NormalTok{ns04a\_02,}\AttributeTok{Tarea1=}\NormalTok{ns03a\_01a,}\AttributeTok{Tarea2=}\NormalTok{ns03a\_01b,}
\AttributeTok{Tarea3=}\NormalTok{ns03a\_01c,}\AttributeTok{Tarea4=}\NormalTok{ns03a\_01d,}\AttributeTok{Tarea5=}\NormalTok{ns03a\_01e,}\AttributeTok{Tarea6=}\NormalTok{ns03a\_01f,}\AttributeTok{Tarea7=}\NormalTok{ns03a\_01g,}
\AttributeTok{Tarea8=}\NormalTok{ns03a\_01h,}\AttributeTok{usual=}\NormalTok{ns03a\_03a)}
\end{Highlighting}
\end{Shaded}

\hypertarget{datos-pre-procesados}{%
\subsubsection{Datos pre-procesados}\label{datos-pre-procesados}}

\hypertarget{transformaciuxf3n}{%
\subsection{Transformación}\label{transformaciuxf3n}}

\begin{Shaded}
\begin{Highlighting}[]
\FunctionTok{library}\NormalTok{(tidyverse)}
\end{Highlighting}
\end{Shaded}

\begin{verbatim}
## Warning: package 'tidyverse' was built under R version 4.2.2
\end{verbatim}

\begin{verbatim}
## -- Attaching packages --------------------------------------- tidyverse 1.3.2 --
## v tibble  3.1.8     v stringr 1.4.1
## v readr   2.1.2     v forcats 0.5.2
## v purrr   0.3.4     
## -- Conflicts ------------------------------------------ tidyverse_conflicts() --
## x dplyr::filter()    masks stats::filter()
## x dplyr::lag()       masks stats::lag()
## x Hmisc::src()       masks dplyr::src()
## x Hmisc::summarize() masks dplyr::summarize()
\end{verbatim}

\begin{Shaded}
\begin{Highlighting}[]
\CommentTok{\#Omitir los datos vacíos o nulos.}
\NormalTok{bd }\OtherTok{\textless{}{-}}\NormalTok{ bd }\SpecialCharTok{\%\textgreater{}\%} \FunctionTok{na.omit}\NormalTok{(bd)}
\end{Highlighting}
\end{Shaded}

\hypertarget{mineruxeda-de-datos}{%
\subsection{Minería de datos}\label{mineruxeda-de-datos}}

\begin{Shaded}
\begin{Highlighting}[]
\DocumentationTok{\#\# Elaboración del modelo}
\FunctionTok{library}\NormalTok{(rpart)}
\FunctionTok{set.seed}\NormalTok{(}\DecValTok{123}\NormalTok{)}
\NormalTok{index }\OtherTok{=} \FunctionTok{sample}\NormalTok{(}\DecValTok{1}\SpecialCharTok{:}\DecValTok{2}\NormalTok{, }\FunctionTok{nrow}\NormalTok{(bd), }\AttributeTok{replace =} \ConstantTok{TRUE}\NormalTok{, }\AttributeTok{prob=}\FunctionTok{c}\NormalTok{(}\FloatTok{0.7}\NormalTok{, }\FloatTok{0.3}\NormalTok{))}
\FunctionTok{prop.table}\NormalTok{(}\FunctionTok{table}\NormalTok{(index))}
\end{Highlighting}
\end{Shaded}

\begin{verbatim}
## index
##         1         2 
## 0.7032219 0.2967781
\end{verbatim}

\begin{Shaded}
\begin{Highlighting}[]
\NormalTok{trainbd}\OtherTok{\textless{}{-}}\NormalTok{bd[index}\SpecialCharTok{==}\DecValTok{1}\NormalTok{,]}
\NormalTok{testbd}\OtherTok{\textless{}{-}}\NormalTok{bd[index}\SpecialCharTok{==}\DecValTok{2}\NormalTok{,]}
\NormalTok{mod1}\OtherTok{\textless{}{-}}\FunctionTok{rpart}\NormalTok{(condac}\SpecialCharTok{\textasciitilde{}}\NormalTok{.,}\AttributeTok{data=}\NormalTok{trainbd)}

\CommentTok{\#Explorar los nodos creados por rpart}
\NormalTok{mod1}
\end{Highlighting}
\end{Shaded}

\begin{verbatim}
## n= 5784 
## 
## node), split, n, loss, yval, (yprob)
##       * denotes terminal node
## 
##  1) root 5784 1308 Sin actividad laboral o trabajo (0.7738589 0.2261411)  
##    2) area=Urbana 4219  560 Sin actividad laboral o trabajo (0.8672671 0.1327329) *
##    3) area=Rural 1565  748 Sin actividad laboral o trabajo (0.5220447 0.4779553)  
##      6) depto=Cochabamba,Oruro,Santa Cruz,Beni,Pando 713  200 Sin actividad laboral o trabajo (0.7194951 0.2805049) *
##      7) depto=Chuquisaca,La Paz,Potosí,Tarija 852  304 Con actividad laboral o trabajo (0.3568075 0.6431925)  
##       14) Tarea7=2.No 331  135 Sin actividad laboral o trabajo (0.5921450 0.4078550)  
##         28) edad< 8.5 113   19 Sin actividad laboral o trabajo (0.8318584 0.1681416) *
##         29) edad>=8.5 218  102 Con actividad laboral o trabajo (0.4678899 0.5321101)  
##           58) depto=Chuquisaca,Potosí,Tarija 154   67 Sin actividad laboral o trabajo (0.5649351 0.4350649) *
##           59) depto=La Paz 64   15 Con actividad laboral o trabajo (0.2343750 0.7656250) *
##       15) Tarea7=1.Si 521  108 Con actividad laboral o trabajo (0.2072937 0.7927063) *
\end{verbatim}

\begin{Shaded}
\begin{Highlighting}[]
\CommentTok{\#Examinar los parámetros del árbol con printcp}
\FunctionTok{printcp}\NormalTok{(mod1)}
\end{Highlighting}
\end{Shaded}

\begin{verbatim}
## 
## Classification tree:
## rpart(formula = condac ~ ., data = trainbd)
## 
## Variables actually used in tree construction:
## [1] area   depto  edad   Tarea7
## 
## Root node error: 1308/5784 = 0.22614
## 
## n= 5784 
## 
##         CP nsplit rel error  xerror     xstd
## 1 0.093272      0   1.00000 1.00000 0.024324
## 2 0.046636      2   0.81346 0.81651 0.022561
## 3 0.012997      3   0.76682 0.78670 0.022236
## 4 0.010000      5   0.74083 0.76988 0.022048
\end{verbatim}

\begin{Shaded}
\begin{Highlighting}[]
\CommentTok{\#Usar el comando plotcp para explorar los parámetros de forma gráfica}
\FunctionTok{plotcp}\NormalTok{(mod1)}
\end{Highlighting}
\end{Shaded}

\includegraphics{ProyectoFinal_MDD_2022_files/figure-latex/unnamed-chunk-4-1.pdf}

\begin{Shaded}
\begin{Highlighting}[]
\CommentTok{\#Usar la función summary para para examinar el modelo}
\FunctionTok{summary}\NormalTok{(mod1)}
\end{Highlighting}
\end{Shaded}

\begin{verbatim}
## Call:
## rpart(formula = condac ~ ., data = trainbd)
##   n= 5784 
## 
##           CP nsplit rel error    xerror       xstd
## 1 0.09327217      0 1.0000000 1.0000000 0.02432355
## 2 0.04663609      2 0.8134557 0.8165138 0.02256060
## 3 0.01299694      3 0.7668196 0.7866972 0.02223623
## 4 0.01000000      5 0.7408257 0.7698777 0.02204806
## 
## Variable importance
##             area           Tarea7            depto             edad 
##               44               23               21                4 
##         inscrito         sabeLeer maxNivelAprobado maxCursoAprobado 
##                3                2                2                1 
##           Tarea5 
##                1 
## 
## Node number 1: 5784 observations,    complexity param=0.09327217
##   predicted class=Sin actividad laboral o trabajo  expected loss=0.2261411  P(node) =1
##     class counts:  4476  1308
##    probabilities: 0.774 0.226 
##   left son=2 (4219 obs) right son=3 (1565 obs)
##   Primary splits:
##       area   splits as  LR,        improve=272.09680, (0 missing)
##       Tarea7 splits as  RL,        improve=203.30490, (0 missing)
##       depto  splits as  RLLLRLLLL, improve= 92.64829, (0 missing)
##       edad   < 10.5 to the left,   improve= 79.09017, (0 missing)
##       Tarea5 splits as  RL,        improve= 70.20063, (0 missing)
##   Surrogate splits:
##       Tarea7   splits as  RL,        agree=0.806, adj=0.282, (0 split)
##       depto    splits as  LLLLRLLLR, agree=0.743, adj=0.051, (0 split)
##       inscrito splits as  LLLR,      agree=0.743, adj=0.049, (0 split)
## 
## Node number 2: 4219 observations
##   predicted class=Sin actividad laboral o trabajo  expected loss=0.1327329  P(node) =0.729426
##     class counts:  3659   560
##    probabilities: 0.867 0.133 
## 
## Node number 3: 1565 observations,    complexity param=0.09327217
##   predicted class=Sin actividad laboral o trabajo  expected loss=0.4779553  P(node) =0.270574
##     class counts:   817   748
##    probabilities: 0.522 0.478 
##   left son=6 (713 obs) right son=7 (852 obs)
##   Primary splits:
##       depto            splits as  RRLLRRLLL, improve=102.11980, (0 missing)
##       Tarea7           splits as  RL, improve= 67.95451, (0 missing)
##       edad             < 8.5  to the left,  improve= 65.32596, (0 missing)
##       Tarea5           splits as  RL, improve= 41.74621, (0 missing)
##       maxNivelAprobado splits as  L-L--RRR-R--------, improve= 40.28798, (0 missing)
##   Surrogate splits:
##       inscrito         splits as  RLLR, agree=0.563, adj=0.041, (0 split)
##       Tarea7           splits as  RL, agree=0.559, adj=0.032, (0 split)
##       asistencia       splits as  RL, agree=0.548, adj=0.007, (0 split)
##       maxNivelAprobado splits as  R-R--RRR-L--------, agree=0.545, adj=0.001, (0 split)
##       usual            splits as  R-L, agree=0.545, adj=0.001, (0 split)
## 
## Node number 6: 713 observations
##   predicted class=Sin actividad laboral o trabajo  expected loss=0.2805049  P(node) =0.1232711
##     class counts:   513   200
##    probabilities: 0.719 0.281 
## 
## Node number 7: 852 observations,    complexity param=0.04663609
##   predicted class=Con actividad laboral o trabajo  expected loss=0.3568075  P(node) =0.1473029
##     class counts:   304   548
##    probabilities: 0.357 0.643 
##   left son=14 (331 obs) right son=15 (521 obs)
##   Primary splits:
##       Tarea7           splits as  RL, improve=59.95731, (0 missing)
##       edad             < 7.5  to the left,  improve=57.13707, (0 missing)
##       sabeLeer         splits as  RL, improve=38.27188, (0 missing)
##       maxNivelAprobado splits as  L-L--RRR----------, improve=35.24853, (0 missing)
##       Tarea5           splits as  RL, improve=32.72622, (0 missing)
##   Surrogate splits:
##       depto            splits as  RR--RL---, agree=0.635, adj=0.060, (0 split)
##       edad             < 6.5  to the left,  agree=0.630, adj=0.048, (0 split)
##       inscrito         splits as  RL-R, agree=0.627, adj=0.039, (0 split)
##       sabeLeer         splits as  RL, agree=0.622, adj=0.027, (0 split)
##       maxNivelAprobado splits as  L-L--RRL----------, agree=0.620, adj=0.021, (0 split)
## 
## Node number 14: 331 observations,    complexity param=0.01299694
##   predicted class=Sin actividad laboral o trabajo  expected loss=0.407855  P(node) =0.05722683
##     class counts:   196   135
##    probabilities: 0.592 0.408 
##   left son=28 (113 obs) right son=29 (218 obs)
##   Primary splits:
##       edad             < 8.5  to the left,  improve=19.718080, (0 missing)
##       Tarea5           splits as  RL, improve=15.752020, (0 missing)
##       maxNivelAprobado splits as  L-L--LRR----------, improve=14.216920, (0 missing)
##       sabeLeer         splits as  RL, improve=11.573830, (0 missing)
##       depto            splits as  LR--LL---, improve= 6.633786, (0 missing)
##   Surrogate splits:
##       maxCursoAprobado < 2.5  to the left,  agree=0.816, adj=0.460, (0 split)
##       sabeLeer         splits as  RL, agree=0.813, adj=0.451, (0 split)
##       maxNivelAprobado splits as  L-L--RRR----------, agree=0.813, adj=0.451, (0 split)
##       Tarea5           splits as  RL, agree=0.776, adj=0.345, (0 split)
##       asistencia       splits as  RL, agree=0.662, adj=0.009, (0 split)
## 
## Node number 15: 521 observations
##   predicted class=Con actividad laboral o trabajo  expected loss=0.2072937  P(node) =0.09007607
##     class counts:   108   413
##    probabilities: 0.207 0.793 
## 
## Node number 28: 113 observations
##   predicted class=Sin actividad laboral o trabajo  expected loss=0.1681416  P(node) =0.01953665
##     class counts:    94    19
##    probabilities: 0.832 0.168 
## 
## Node number 29: 218 observations,    complexity param=0.01299694
##   predicted class=Con actividad laboral o trabajo  expected loss=0.4678899  P(node) =0.03769018
##     class counts:   102   116
##    probabilities: 0.468 0.532 
##   left son=58 (154 obs) right son=59 (64 obs)
##   Primary splits:
##       depto            splits as  LR--LL---, improve=9.880410, (0 missing)
##       Tarea5           splits as  RL, improve=6.210742, (0 missing)
##       edad             < 12.5 to the left,  improve=4.606581, (0 missing)
##       maxNivelAprobado splits as  -----LRR----------, improve=2.855229, (0 missing)
##       Tarea8           splits as  LR, improve=2.521765, (0 missing)
##   Surrogate splits:
##       maxNivelAprobado splits as  -----LLR----------, agree=0.711, adj=0.016, (0 split)
## 
## Node number 58: 154 observations
##   predicted class=Sin actividad laboral o trabajo  expected loss=0.4350649  P(node) =0.02662517
##     class counts:    87    67
##    probabilities: 0.565 0.435 
## 
## Node number 59: 64 observations
##   predicted class=Con actividad laboral o trabajo  expected loss=0.234375  P(node) =0.01106501
##     class counts:    15    49
##    probabilities: 0.234 0.766
\end{verbatim}

\begin{Shaded}
\begin{Highlighting}[]
\CommentTok{\#Visualizar el árbol}
\CommentTok{\#install.packages("rpart.plot")}
\FunctionTok{library}\NormalTok{(rpart.plot)}
\end{Highlighting}
\end{Shaded}

\begin{verbatim}
## Warning: package 'rpart.plot' was built under R version 4.2.2
\end{verbatim}

\begin{Shaded}
\begin{Highlighting}[]
\FunctionTok{rpart.plot}\NormalTok{(mod1)}
\end{Highlighting}
\end{Shaded}

\includegraphics{ProyectoFinal_MDD_2022_files/figure-latex/unnamed-chunk-4-2.pdf}
\#\# Interpretación

\begin{Shaded}
\begin{Highlighting}[]
\NormalTok{clase}\OtherTok{\textless{}{-}}\FunctionTok{predict}\NormalTok{(mod1,testbd,}\AttributeTok{type =} \StringTok{"class"}\NormalTok{)}
\FunctionTok{table}\NormalTok{(clase,testbd}\SpecialCharTok{$}\NormalTok{condac)}
\end{Highlighting}
\end{Shaded}

\begin{verbatim}
##                                  
## clase                             Sin actividad laboral o trabajo
##   Sin actividad laboral o trabajo                            1869
##   Con actividad laboral o trabajo                              66
##                                  
## clase                             Con actividad laboral o trabajo
##   Sin actividad laboral o trabajo                             316
##   Con actividad laboral o trabajo                             190
\end{verbatim}

\begin{Shaded}
\begin{Highlighting}[]
\CommentTok{\#install.packages(\textquotesingle{}caret\textquotesingle{})}
\FunctionTok{library}\NormalTok{(caret)}
\end{Highlighting}
\end{Shaded}

\begin{verbatim}
## Warning: package 'caret' was built under R version 4.2.2
\end{verbatim}

\begin{verbatim}
## 
## Attaching package: 'caret'
\end{verbatim}

\begin{verbatim}
## The following object is masked from 'package:purrr':
## 
##     lift
\end{verbatim}

\begin{verbatim}
## The following object is masked from 'package:survival':
## 
##     cluster
\end{verbatim}

\begin{Shaded}
\begin{Highlighting}[]
\FunctionTok{confusionMatrix}\NormalTok{(}\FunctionTok{table}\NormalTok{(clase,testbd}\SpecialCharTok{$}\NormalTok{condac))}
\end{Highlighting}
\end{Shaded}

\begin{verbatim}
## Confusion Matrix and Statistics
## 
##                                  
## clase                             Sin actividad laboral o trabajo
##   Sin actividad laboral o trabajo                            1869
##   Con actividad laboral o trabajo                              66
##                                  
## clase                             Con actividad laboral o trabajo
##   Sin actividad laboral o trabajo                             316
##   Con actividad laboral o trabajo                             190
##                                                          
##                Accuracy : 0.8435                         
##                  95% CI : (0.8285, 0.8577)               
##     No Information Rate : 0.7927                         
##     P-Value [Acc > NIR] : 9.343e-11                      
##                                                          
##                   Kappa : 0.4176                         
##                                                          
##  Mcnemar's Test P-Value : < 2.2e-16                      
##                                                          
##             Sensitivity : 0.9659                         
##             Specificity : 0.3755                         
##          Pos Pred Value : 0.8554                         
##          Neg Pred Value : 0.7422                         
##              Prevalence : 0.7927                         
##          Detection Rate : 0.7657                         
##    Detection Prevalence : 0.8951                         
##       Balanced Accuracy : 0.6707                         
##                                                          
##        'Positive' Class : Sin actividad laboral o trabajo
## 
\end{verbatim}

\hypertarget{resultados-y-anuxe1lisis}{%
\section{Resultados y análisis}\label{resultados-y-anuxe1lisis}}

\hypertarget{conclusiones-y-recomendaciones}{%
\section{Conclusiones y
recomendaciones}\label{conclusiones-y-recomendaciones}}

\hypertarget{referencias}{%
\section{Referencias}\label{referencias}}

\end{document}
